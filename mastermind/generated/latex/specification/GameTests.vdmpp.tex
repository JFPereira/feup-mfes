\begin{vdmpp}[breaklines=true]
class GameTests is subclass of MyTestCase -- represents the test suite

operations

 -- check if a key is correctly created
(*@
\label{testCreateKey:6}
@*)
 public testCreateKey : () ==> ()
  testCreateKey() == (
   dcl key : Player`Key := [1, 2, 3, 4];
   
   assertEqual([1, 2, 3, 4], key);
   assertEqual(4, len key);
   
   for x in key do ( 
    assertTrue(x >= 1 and x <= 6);
   )
  );
  
 -- check if a key is correctly created
(*@
\label{testCreateRandomlyKey:19}
@*)
 public testCreateRandomlyKey : () ==> ()
  testCreateRandomlyKey() == (
   dcl key : Player`Key := Player`GenerateRandomlyKey();
   
   assertEqual(4, len key);
   for x in key do ( 
    assertTrue(x >= 1 and x <= 6);
   )
  );
  
  -- check if a key can be created with five elements
(*@
\label{testCreateKeyWithMoreThanFourElems:30}
@*)
 public testCreateKeyWithMoreThanFourElems : () ==> ()
  testCreateKeyWithMoreThanFourElems() == (
   dcl key : Player`Key;
   
   key := [1, 2, 3, 4, 5];
  );

 -- checks if a default player is created with a certain key
(*@
\label{testPlayer:38}
@*)
 public testPlayer : () ==> ()
  testPlayer() == (
  dcl player: Player := new Player();
  player.SetKey([1, 2, 3, 4]);
  assertEqual([1, 2, 3, 4], player.GetKey());
  assertEqual("Default", player.GetName());
  );
 
 -- checks if a player is created with a certain key
(*@
\label{testPlayerWithKey:47}
@*)
 public testPlayerWithKey : () ==> ()
  testPlayerWithKey() == (
  dcl player: Player := new Player("Cristo", [1, 2, 3, 4]);
  assertEqual([1, 2, 3, 4], player.GetKey());
  );
  
 -- check if the sequence returns a correct value
(*@
\label{testGameMove:54}
@*)
 public testGameMove : () ==> ()
  testGameMove() == (
   dcl codeMaker: Player := new Player("CodeMaker", [6, 6, 5, 2]);
    dcl codeBreaker: Player := new Player("CodeBreaker");
    dcl g: Game := new Game(codeMaker, codeBreaker);
    
    -- the result of the game initial moves is 10
    assertEqual(10, g.GetCurrentMoves());
    
    -- code breaker makes the move [1, 1, 2, 2], and the expected result is (1, 0)
    g.MakeMove([1, 1, 2, 2]);
    assertEqual([1, 0], g.GetLastResult());
  );
  
 -- check if a invalid move is rejected
(*@
\label{testInvalidGameMove:69}
@*)
 public testInvalidGameMove : () ==> ()
  testInvalidGameMove() == (
   dcl codeMaker: Player := new Player("CodeMaker", [6, 6, 5, 2]);
    dcl codeBreaker: Player := new Player("CodeBreaker");
    dcl g: Game := new Game(codeMaker, codeBreaker);
    
    -- the result of the game initial moves is 10
    assertEqual(10, g.GetCurrentMoves());
    
    -- code breaker makes the move [1, 1, 7, 2], and it should be rejected
    g.MakeMove([1, 1, 7, 2]);
  );

 -- simulates a game with the code breaker as the winner
(*@
\label{testGameWinnerCodeBreaker:83}
@*)
  public testGameWinnerCodeBreaker : () ==> ()
   testGameWinnerCodeBreaker() == (
    -- create game players
    dcl codeMaker: Player := new Player("CodeMaker", [6, 6, 5, 2]);
    dcl codeBreaker: Player := new Player("CodeBreaker");
    
    -- create game
    dcl g: Game := new Game(codeMaker, codeBreaker);
    
    -- the result of the game initial moves is 10
    assertEqual(10, g.GetCurrentMoves());
    
    -- code breaker makes the move [1, 1, 2, 2], and the expected result is (1, 0)
    g.MakeMove([1, 1, 2, 2]);
    assertEqual([1, 0], g.GetLastResult());
    
    -- code breaker makes the move [3, 3, 4, 4], and the expected result is (0, 0)
    g.MakeMove([3, 3, 4, 4]);
    assertEqual([0, 0], g.GetLastResult());
    
    -- code breaker makes the move [5, 5, 6, 6], and the expected result is (0, 3)
    g.MakeMove([5, 5, 6, 6]);
    assertEqual([0, 3], g.GetLastResult());
    
    -- code breaker makes the move [5, 6, 5, 2], and the expected result is (3, 0)
    g.MakeMove([5, 6, 5, 2]);
    assertEqual([3, 0], g.GetLastResult());
    
    -- code breaker makes the move [6, 6, 5, 2], the expected result is (4, 0) and he wins the game
    g.MakeMove([6, 6, 5, 2]);
    assertEqual([4, 0], g.GetLastResult());
    
    -- the winner was the code breaker
    assertEqual("CodeBreaker", g.GetWinnerPlayer().GetName());
    
    IO`print("Winner is: ");
    IO`println(g.GetWinnerPlayer().GetName());
   );
   
   -- simulates a game with the code maker as the winner
(*@
\label{testGameWinnerCodeMaker:123}
@*)
   public testGameWinnerCodeMaker : () ==> ()
   testGameWinnerCodeMaker() == (
    -- create game players
    dcl codeMaker: Player := new Player("CodeMaker", [1, 2, 4, 1]);
    dcl codeBreaker: Player := new Player("CodeBreaker");
    
    -- create game
    dcl g: Game := new Game(codeMaker, codeBreaker);
    
    -- the result of the game initial moves is 10
    assertEqual(10, g.GetCurrentMoves());
    
    -- code breaker makes the move [1, 1, 2, 2], and the expected result is (1, 2)
    g.MakeMove([1, 1, 2, 2]);
    assertEqual([1, 2], g.GetLastResult());
    
    -- code breaker makes the move [3, 3, 4, 4], and the expected result is (1, 0)
    g.MakeMove([3, 3, 4, 4]);
    assertEqual([1, 0], g.GetLastResult());
    
    -- code breaker makes the move [5, 5, 6, 6], and the expected result is (0, 0)
    g.MakeMove([5, 5, 6, 6]);
    assertEqual([0, 0], g.GetLastResult());
    
    -- code breaker makes the move [2, 1, 3, 4], and the expected result is (0, 3)
    g.MakeMove([2, 1, 3, 4]);
    assertEqual([0, 3], g.GetLastResult());
    
    -- code breaker makes the move [1, 1, 1, 1], and the expected result is (2, 0)
    g.MakeMove([1, 1, 1, 1]);
    assertEqual([2, 0], g.GetLastResult());
    
    -- code breaker makes the move [2, 2, 2, 2], and the expected result is (1, 0)
    g.MakeMove([2, 2, 2, 2]);
    assertEqual([1, 0], g.GetLastResult());
    
    -- code breaker makes the move [1, 1, 3, 4], and the expected result is (1, 2)
    g.MakeMove([1, 1, 3, 4]);
    assertEqual([1, 2], g.GetLastResult());
    
    -- code breaker makes the move [2, 1, 3, 4], and the expected result is (0, 3)
    g.MakeMove([2, 1, 3, 4]);
    assertEqual([0, 3], g.GetLastResult());
    
    -- code breaker makes the move [1, 1, 2, 4], and the expected result is (1, 3)
    g.MakeMove([1, 1, 2, 4]);
    assertEqual([1, 3], g.GetLastResult());
    
    -- code breaker makes the move [1, 2, 1, 4], the expected result is (2, 2) and he wins the game
    g.MakeMove([1, 2, 1, 4]);
    assertEqual([2, 2], g.GetLastResult());
    
    -- the winner was the code breaker
    assertEqual("CodeMaker", g.GetWinnerPlayer().GetName());
    
    IO`print("Winner is: ");
    IO`println(g.GetWinnerPlayer().GetName());
   );
  
  -- simulates a game with randomly moves
(*@
\label{testRandomlyGame:183}
@*)
  public testRandomlyGame : () ==> ()
   testRandomlyGame() == (
    dcl codeMaker: Player := new Player("CodeMaker", [1, 2, 4, 1]);
    dcl codeBreaker: Player := new Player("CodeBreaker");
    dcl g: Game := new Game(codeMaker, codeBreaker);
    
    -- the result of the game initial moves is 10
    assertEqual(10, g.GetCurrentMoves());
    
    -- play the game with randomly moves
    g.PlayRandomly();
    
    -- the game need to be finished
    assertEqual(true, g.IsFinished());
    
    IO`print("Winner is: ");
    IO`println(g.GetWinnerPlayer().GetName());
   );
  
  -- simulates a championship, where four players play randomly 
(*@
\label{testChampionshipWithFourPlayers:203}
@*)
  public testChampionshipWithFourPlayers : () ==> ()
   testChampionshipWithFourPlayers() == (
    -- creating four players to the championship
    dcl p1 : Player := new Player("Tito");
    dcl p2 : Player := new Player("Cristina");
    dcl p3 : Player := new Player("Jacinto");
    dcl p4 : Player := new Player("Ana");
    dcl games : seq of Game;
    
    -- creating a championship
    dcl champ : Championship := new Championship([p1, p2, p3, p4]);
    assertEqual(4, len champ.GetPlayers());
    
    -- run championship by rounds
    for i = 1 to champ.GetNumberOfRounds() by 1 do (
     -- creates a championship round with the selected players
     games := champ.CreateGames(champ.GetCurrentPlayersOnChampionship());
     if i = 1
      then (
       assertEqual(2, len games);
      )
     elseif i = 2
      then (
       assertEqual(1, len games);
      );
     
     for j = 1 to len games by 1 do (
      -- play games randomly
      games(j).PlayRandomly();
     );
     
     -- get all the winner players of the championship round
     champ.PickAllWinnerPlayers(games);
     
     if i = 1
      then (
       assertEqual(2, len champ.GetCurrentPlayersOnChampionship());
      )
     elseif i = 2
      then (
       assertEqual(1, len champ.GetCurrentPlayersOnChampionship());
      );
     
     for j = 1 to len champ.GetCurrentPlayersOnChampionship() by 1 do (
      IO`print("Winner of Game ");
      IO`print(j);
      IO`print(" is: ");
      IO`println(champ.GetCurrentPlayersOnChampionship()(j).GetName());
     );
     
     -- adding games to the championship history
     champ.AddGames(games);
     
    );
     -- prints the total moves of the championship winner player 
     champ.PrintStats();
   );
   
  -- simulates a championship with an odd number of players
(*@
\label{testChampionshipWithOddNumberOfPlayers:262}
@*)
  public testChampionshipWithOddNumberOfPlayers : () ==> ()
   testChampionshipWithOddNumberOfPlayers() == (
    -- creating four players to the championship
    dcl p1 : Player := new Player("Tito");
    dcl p2 : Player := new Player("Cristina");
    dcl p3 : Player := new Player("Jacinto");
    dcl p4 : Player := new Player("Ana");
    dcl p5 : Player := new Player("Antonio");
    
    dcl champ : Championship;
    
    -- creating a championship
    champ := new Championship([p1, p2, p3, p4, p5]);
   );
  
(*@
\label{testAll:277}
@*)
  public testAll: () ==> ()
  testAll() == (
   --testCreateKey();
   --testCreateRandomlyKey();
   --testCreateKeyWithMoreThanFourElems();
    --testPlayer();
    --testPlayerWithKey();
    --testGameMove();
    --testInvalidGameMove();
    --testGameWinnerCodeBreaker();
    --testGameWinnerCodeMaker();
    --testRandomlyGame();
    --testChampionshipWithFourPlayers();
    testChampionshipWithOddNumberOfPlayers();
  );
  
end GameTests
\end{vdmpp}
