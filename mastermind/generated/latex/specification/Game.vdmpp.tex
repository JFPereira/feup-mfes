\begin{vdmpp}[breaklines=true]
class Game -- represents a game between two players

types
 public Result = seq of nat
 inv r == len r = 2 and r(1) + r(2) <= 4;
  
instance variables
 private moves : nat := 10;
 private codeMaker: Player;
 private codeBreaker: Player;
 private makerResults : seq of Result := [];
 private breakerMoves: seq of Player`Key := [];
 private finished : bool := false;

inv len makerResults = 10 - moves;
inv finished => len makerResults > 0 and len makerResults < 11;
inv forall i in set inds makerResults \ {len makerResults} & makerResults(i) <> [4, 0];
        
operations
  -- constructor, initializes the game
(*@
\label{Game:21}
@*)
  public Game: Player * Player ==> Game
   Game(codeMakerPlayer, codeBreakerPlayer) == (
    codeMaker := codeMakerPlayer;
    codeBreaker := codeBreakerPlayer;
    return self
   );
   
  -- get the current number of moves remaining
(*@
\label{GetCurrentMoves:29}
@*)
  pure public GetCurrentMoves : () ==> nat
   GetCurrentMoves () == return moves;
  
  -- get the code maker final key
(*@
\label{GetFinalKey:33}
@*)
  pure public GetFinalKey : () ==> seq of int
   GetFinalKey () == return codeMaker.GetKey();
   
  -- get the code maker player
(*@
\label{GetCodeMaker:37}
@*)
  pure public GetCodeMaker : () ==> Player
   GetCodeMaker () == return codeMaker;
   
  -- get the code breaker player
(*@
\label{GetCodeBreaker:41}
@*)
  pure public GetCodeBreaker : () ==> Player
   GetCodeBreaker () == return codeBreaker;
   
  -- return maker results
(*@
\label{GetMakerResults:45}
@*)
  pure public GetMakerResults : () ==> seq of Result
   GetMakerResults () == return makerResults;
  
  -- return breaker moves 
(*@
\label{GetBreakerMoves:49}
@*)
  pure public GetBreakerMoves : () ==> seq of Player`Key
   GetBreakerMoves () == return breakerMoves;
  
  -- get last result of the maker moves
(*@
\label{GetLastResult:53}
@*)
  pure public GetLastResult : () ==> Result
   GetLastResult() == return makerResults(len makerResults);
  
  -- return the winner player 
(*@
\label{GetWinnerPlayer:57}
@*)
  pure public GetWinnerPlayer : () ==> Player
   GetWinnerPlayer() == (
    if GetLastResult() = [4, 0]
     then
      return codeBreaker
    else
     return codeMaker;
   );
   
  -- tell if a game is finished or not
(*@
\label{IsFinished:67}
@*)
  pure public IsFinished : () ==> bool
   IsFinished() == (
    return finished;
   );
   
  -- adds a new key to the set of the breaker moves and sets the current key of the breaker player
(*@
\label{AddKey:73}
@*)
  public AddKey : Player`Key ==> ()
   AddKey (key) == (
    codeBreaker.SetKey(key);
    breakerMoves := breakerMoves^[key];
   )
  pre len key > 0 and len key < 5 and forall x in set elems key & (x > 0 and x < 7)
  post len breakerMoves = len breakerMoves~ + 1;
  
  -- checks the move of the breaker player
(*@
\label{CheckMove:82}
@*)
  private CheckMove : () ==> ()
  CheckMove () == (
   dcl e_corrects : nat := 0;
   dcl e_exists : nat := 0;
   dcl makerElemRep : nat := 0;
   dcl breakerElemRep : nat := 0;
   dcl sumMins : nat := 0;
   dcl breakerKey : Player`Key := codeBreaker.GetKey();
   dcl makerKey : Player`Key := codeMaker.GetKey();
  
   for i = 1 to len codeBreaker.GetKey() by 1 do (
     if breakerKey(i) = makerKey(i)
      then
       e_corrects := e_corrects + 1;
   );
   
   for i = 1 to 6 by 1 do (
    makerElemRep := Utilities`Count(makerKey, i);
    breakerElemRep := Utilities`Count(breakerKey, i);
    sumMins := sumMins + Utilities`Min(makerElemRep, breakerElemRep);
   );
   
   e_exists := sumMins - e_corrects;
   
   atomic (
    moves := moves - 1;
    makerResults := makerResults^[[e_corrects, e_exists]];
    finished := e_corrects = 4
   ); 
  );
   
  -- print board
(*@
\label{PrintBoard:114}
@*)
  private PrintBoard : seq of Player`Key * seq of Result ==> ()
   PrintBoard (breakerKeys, makerResponses) == (
    dcl currentMove : nat := 10 - moves;
   
    IO`print("Board Game - Move: ");
    IO`println(currentMove);
    IO`println("MOVE                 RESULT [Corrects, Exists]");
    
    for i = len makerResponses to 1 by -1 do (
     IO`print(breakerKeys(i));
     IO`print("         ");
     IO`println(makerResponses(i));
    );
    
    IO`print("Moves Remaing: ");
    IO`println(moves);
    IO`print("\n");
   );
   
  -- make a move
(*@
\label{MakeMove:134}
@*)
  public MakeMove : (Player`Key) ==> ()
   MakeMove (key) == (
   if moves > 0
    then (
     if not finished
      then (
       AddKey(key);
       CheckMove();
       
       PrintBoard(breakerMoves, makerResults);
       
       if finished
        then (
         IO`print(codeBreaker.GetName());
         IO`println(" won the game.");
         IO`print("The key was ");
         IO`print(codeBreaker.GetKey());
         IO`print(" and the number of tries was ");
         IO`print(len makerResults);
         IO`println(".");
         IO`print("\n");
        )
       elseif moves = 0 and not finished 
        then (
         finished := true;
         IO`print(codeMaker.GetName());
         IO`println(" won the game.");
         IO`print("The key was ");
         IO`print(GetFinalKey());
         IO`print(" and the number of moves was ");
         IO`print(len makerResults);
         IO`println(".");
         IO`print("\n");
        );
      )
     else (
      IO`println("The game is over.");
      IO`print("\n");
     )
    )
   else (
    IO`println("The game is over.");
    IO`print("\n");
   )
  )
  pre len key > 0 and len key < 5 and forall x in set elems key & (x > 0 and x < 7);
  
  -- make a full random game
(*@
\label{PlayRandomly:182}
@*)
  public PlayRandomly : () ==> ()
   PlayRandomly() == (
    if moves = 10
     then
      codeMaker.SetKey(Player`GenerateRandomlyKey());
   
    while not finished do (
     MakeMove(Player`GenerateRandomlyKey());
    );
   );
   
end Game
\end{vdmpp}
